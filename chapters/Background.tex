Odense Municipality (OM) is the third greatest in Denmark with an size of 304 $km^2$ and a population of nearly 200.000 people. In order to reach the goals of the national environmental policy, OM has created the directive “Environmentally friendly construction - requirements and recommendations\footnote{See appendix: Appendix 01 -Environmentally friendly construction}”,  which as the directive states will: “\emph{Concretise the environmental policy and objectives, so that future building projects in the municipality  to the up-filling of the overall vision of Odense as Denmark's most sustainable city.}”\footnote{See appendix: Appendix 01 -Environmentally friendly construction}. This directive sets energy efficiency requirements for newly constructed and renovated buildings.

OM owns around 850 buildings, which are used for different purposes like schooling, administration, care centers and cultural institutions. The municipality is responsible for paying suppliers of water, heat and electricity for the consumption in these buildings. This is done on basis of billing meters which measure the consumption, and are installed in each building. Depending on the size of the building, there might be a number of internal meters (so called distribution meters), which measure the consumption in different parts of the building, but are not used as billing meters. 

Each public building has an appointed energy manager who has the responsibility of performing the manual tasks involved in reading and checking the meters. The energy manager logs reading internally in a control book or scheme and reports them on the last weekday of the month through a web portal. If reading deviate more than +/- 10 \% from the budget forecasts, the manager must submit possible causes for this. As the document “Energy efficiency in the municipality of Odense” states, both types of meters must be often checked to avoid energy leakages.  In buildings less than 1.000 $m^2$, meters must be checked at least once a week and for larger buildings the frequency is at least once a day. OM has started installing meters which automatically report their readings on an hourly basis. Those are not yet installed in all buildings, but OM has plans to do so. Depending on the concrete building and setup there is a delay from around 1 to 7 days before readings from the automatic meters are visible in OM’s energy management system ‘Energy Key’\footnote{http://www.emtnordic.com/da/energykey-energistyringsprogram}.

\section{Current limitations}
Through interviews with a representative from OM’s Energy and Maintenance department, the needs for a building energy management system have been investigated. This analysis revealed a number of areas in which improvements could be made. Before 2013 each building's energy consumption was monitored on site by appropriate staff, and energy bills were paid from the budgets of the organization residing in the building. However, building managers on site had no real tools or means of monitoring their consumption, making it difficult to identify high usage or leakages. This function has now been centralized in the department of Energy and Maintenance which owns the buildings and is responsible for monitoring and paying for energy consumption, as well as renovation and maintenance of the buildings. The organizations residing in the buildings will then just rent the building for a fixed price. This centralization should give OM a better control and overview of the municipality’s energy consumption, but a number of new issues have arised. The task has scaled with the number of buildings that now needs to be monitored, and the problem analysis showed that OM’s Energy and Maintenance department does not have the needed staff nor the proper tools to complete it effectively. The problem analysis resulted in the identification of 5 main issues, which has been the focus points of this project.
\section*{System and data maintenance}
Maintaining accurate data about a building's size, age, type, associated meters etc. can seem like a seemingly small effort, but has scaled to become an overwhelming task when considering the amount of buildings. If the primary data is imprecise, erroneous or inconsistent, the value of data analysis and anything that depends on it will decline. OM’s current energy management system Energy Key has no way of telling the user when certain data has last been verified and by whom, nor is there any easy way for users to update incorrect data entries. This makes it difficult to work with the data, because the users are often unsure about the validity of it, and it causes redundant work because multiple users need to verify the same data. There are currently no procedures for how often a building's consumption should be checked and validated, nor does Energy Key support such continuous maintenance tasks in any way. Additionally OM struggles with poor data quality and many missing values regarding consumption figures, which according to OM is because the energy suppliers’ systems are simply not built for sharing data with clients.
\section*{Visualization of buildings and consumptions}
Energy Key allows users to view electricity, water and heat consumption data for individual meters, but the possibilities to visualize and represent data in meaningful formats seems limited. The user can view diagrams depicting consumption over time, and histograms that compare a building’s reported consumption with the budget. Other than this, the system does not summarize important key features and energy statistics for a given building. It does not analyze the data to provide new insights nor does it provide any tools for users to do so. The lack of meaningful information and visualization of consumption patterns makes it hard to understand a building’s energy usage. This makes it more difficult to motivate users and building managers into making better decisions that lead to energy savings, because they do not know how their behaviour influences the consumption.
\section*{Fault detection, diagnosis and resolution}
 Energy Key has an automatic fault detection feature which can raise an alarm when consumption goes above or below a given percentage of what is budgeted. OM’s Energy and Maintenance department has tried experimenting with this system for a single building, and the results were that 37 false alarms were received in 2012 and 34 in 2013. When considering all buildings the amount of false alarms becomes overwhelming. The large amount of false alarms combined with the fact that tasks involved in diagnosis and resolution of faults are not supported in Energy Key, has lead to the alarms being completely ignored. Even if a user can verify that an alarm is valid the fault might be neglected, as there are no standard procedures to follow in order to resolve the fault, nor is there any personnel assigned as responsible. Energy Key simply does not support users in coordinating their efforts towards the resolution of alarms.
 \section*{Budgeting}
 Currently energy consumption and budgets for each building are estimated by looking at the same building’s consumption in previous years. Estimations have monthly granularities, meaning that the energy consumption estimate for a given month depends on the consumption reported the same month previous years. This approach does not allow one to properly predict consumption under changing circumstances (people usage, environment, building changes etc.). A better model for a building’s energy consumption would allow for better predictions and more exact budgets. When information about a building is changed in Energy Key, the previous information is lost making it is difficult to use the previous consumption patterns to plan future budgets as there is no historic data to show under which conditions the previous consumption was produced. Because of the issues with maintaining data in Energy Key, information like for instance how many people are using a building is stored and maintained in excel files.
 \section*{Selecting buildings for renovation}
 In 2015 OM’s Energy and Maintenance department has budgeted 200 million dkk to renovation of public buildings. As the \emph{directive Environmentally friendly construction - requirements and recommendations}\footnote{See Appendix 01 -Environmentally friendly construction} says all renovations must be done with the most energy efficient solutions. The department now faces the task of finding the set of buildings for which renovations would bring down energy usage the most. Currently a taskforce is screening through the available data manually, and more or less takes educated guesses about which buildings are inefficient and can be optimized. The current system is unable to perform any sophisticated benchmarking or ranking of buildings, making it hard to verify that the most energy inefficient buildings have been found. However, identifying energy inefficient buildings is only the starting point, as a vast range of economical and political parameters also have to be considered before a building is finally selected for renovation.