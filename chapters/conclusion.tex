To conclude whatever the project was successful, the developed systems capabilities will be reviewed against the project initial goals. In the introduction they are stated as being, to create a functioning prototype website capable of highlighting various trends, correlations and odd events in respect to energy consumption in buildings. Furthermore it was a goal to create a system that was tailored to the needs of the Municipality of Odense. This made usability and practical application of the system a requirement in order to make it benefit people in the real world. In relation to this, the developed systems ability to solve the municipality?s five issues which were identified and described in section \emph{Background}, will be used to conclude whether this project has been successful.
\section{System and data maintenance}
With OM?s current system they do not have the resources to focus their work on becoming more energy efficient, because system and data maintenance have become overwhelming tasks. The developed system counters this problem by being fully integrated and automated with the existing system. The system uses database structures which ensure that the data is always consistent, and it actively supports and encourages users to validate that old data is still valid. This functionality ensures that redundant work in validating data is avoided. With these features, BEOVulf comes a long way in order to lessen the tasks of maintaining system and data, thereby releasing resources which can be allocated for optimizing the energy efficiency.
\section{Visualization of buildings and consumptions}
As for the second issue, the municipality lacks the tools needed in order to understand consumptions and manage the buildings, the system provides the following tools:
\begin{itemize}
\item Visualization of raw consumption.
\item Visualization of various energy consumption patterns.
\item Visualization of key consumption features including seasonality, nightly consumption, peak loads and more.
\end{itemize}
Compared to the municipality's current system, these features considerably lessen the effort involved in getting useful information about a building?s energy consumption. This will make OM?s Energy and Maintenance department more effective and better enabled to make informed decisions. In this respect BEOVulf has met the goals for visualization of buildings and consumptions.
\section{Fault detection, diagnosis and resolution}
Due to limited resources OM?s Energy and Maintenance department can not manually inspect and monitor all buildings, but rather need to focus their effort on the buildings that need attention. BEOVulfs fault detection system has proven that it can detect anomalies in consumption patterns, and thus it is able to notify energy managers when a building needs attention. Thus OM?s Energy and Maintenance department can focus their work on correcting the faults instead of on searching for the needle in the haystack.

The automatic fault detection will enable OM to respond faster to leakages and thereby minimize energy and water waste. The system is able to suggest which aspects of the consumption that triggered the fault, and even though it is not fully implemented in the prototype, the fault detection system is also aimed to be a task management system. This enables OM to effectively assign and prioritize resources towards fault resolution.
\section{Budgeting}
In relation to the fourth issue of creating better and more accurate budgets, the visualization tools provides the means for estimating more precisely the consumption of a building. With these tools the department of Energy and Maintenance will no longer have to rely on a old unvalidated data or as it was the case sometimes, a professional gues. However, BEOVulf does not provide any tools to directly support budgeting tasks. This was part of the initial aim, but was within the limited time of the project prioritized lower in favor of other requirements.
\section{Selecting buildings for renovation}
The task of selecting buildings for renovation involves comparing and screening the consumption of many buildings. This is a difficult task to do with OM?s current system. To make this task easy BEOVulf implements functionality enabling the user to quickly compare many different features for different buildings, by having them plotted in the same graph. The system also proved that it performs well in automatically identifying similar buildings, allowing the user to browse through them and visualize their consumption. These features all help users generate an initial set of candidate buildings for renovation. However, the development of the system did not come as far as having a feature that automatically identifies these candidates, which was originally the goal.
\section{The Project}
Many improvements are still to be made if BEOVulf is to become a complete system that fully fulfils the project’s initial goals. However as a fully functional prototype it comes a long way. BEOVulf is able to visualize consumption, detect anomalies, find similar buildings and a user test showed that the system is practically usable. Thus the conclusion is that BEOVulf successfully applies data mining and statistical methods to recognize and visualize energy consumption patterns of buildings.