Based on the initial problem analysis and identification of OM’s current limitations, a quick low fidelity horizontal prototype system was created\footnote{See Appendix 04 -Website Alpha Prototype.zip}. This prototype was evaluated through an interview with a representative from OM’s Energy and Maintenance department\footnote{See Appendix 02 -Meeting with Kim Allan 29-4-15}. Based on this interview the problem analysis was deepened and a final backlog of requirements was for the system to be developed was specified. The full requirements list can be found in Appendix 05 -Requirement Specification, while this section will give an outline of how the requirements solve the identified issues. 
\section{System and data maintenance}
Any client of a new software system would require that the servicing efforts are minimal, but the problem analysis uncovered that for OM this is of critical importance. OM’s Energy and Maintenance department has had a large increase in workload without a proportional increase in resources or personnel. This means that the department simply does not have the time or resources to maintain a new system if the required servicing effort is too high. A number of steps to prevent this have been taken, resulting in formal requirements which can be measured and evaluated. 

Firstly, in long term, the new system should replace OM’s existing system, but changes in an organization's processes, workflow and technology comes with the expense of time and money. In order to minimize the expenses, a gradual adoption of the new system will be in focus. Requirement R18, ensures data consistency in a such adoption phase where multiple systems work on the same data. This ensures further, that maintenance in an adoption phase will remain the same as in the existing system.

Secondly, to minimize the effort needed to maintain large amounts of building information (such as type, age size and so forth), requirement F2-06a and F2-06b make sure it is possible to easily update and validate these general building informations. The effort associated with maintaining the data is minimized by informing the user when data was last validated and if it is time to re-validate. Thus the user needs not spend time investigating validity of data every time it is needed. Also as opposed to the current system it will be possible to easily update data when a building is viewed anyway, and then offering the user the choice to validate it now. This solution ensures that even building managers for the different buildings will participate in keeping the information correct. This is additionally supported with requirement F15 and F17, which will allow the assignment of maintenance tasks to the building managers thus distributing the maintenance effort.
\section{Visualization of buildings and consumptions}
Requirements F2 and F9 together with the respective sub requirements state that it should be possible to view and compare information about buildings and their energy consumption. The requirements formulate a number of concrete aspects of the energy consumption that must be visualized to the user. In contrast to OM’s existing system this will give users better information and support better decision making. Requirement F2-02 gives concrete examples of a number of energy features that must be extracted and visualized from the raw consumption numbers. Those are for instance peak loads, standby usage and daily average consumptions.

Besides pure consumption figures, the visualization of consumption patterns can further enhance a manager's understanding of the buildings consumption. The consumption patterns of a building may be viewed in different levels of granularity, like for instance patterns on a micro scale (dayly) and patterns on a macro scale (yearly). Depending on the user’s task, he might be interested in patterns on different scales. The three requirements F2-03a to F2-03c  describe how consumption patterns on a micro scale is to be visualized through so called daily consumption profiles. Requirement F2-04 requires that a buildings seasonal variation in consumption is visualized which might be considered a macro scale pattern. 

From looking only at the consumption of a single building without a context or reference point, it can be hard to identify faults or even verify the that everything is as it should be. Therefore requirement F9 and its sub requirements aim to give the user a way to visualize a building in comparison with others, to have a reference point when evaluating a building\footnote{Requirement F9 and comparing buildings is further discussed in the Selecting of buildings for renovation section}. 
\section{Fault detection, diagnosis and resolution}
Requirement F3, addresses the fault detection, and here the aim is to build a fault detection system which is more accurate and allows for diagnosing the feature which caused the flaw. 

Requirements F3-02, F3-03 and F12 (handling of alarms) will try to make sure alarms are not ignored or forgotten by establishing a process/workflow for how they should be handled. The system will provide an easy overview of alarms, and requirements regarding task management (F15-17) will contribute to the creation of a “standard” procedure of handling alarms.
\section{Budgeting}
Requirement F14 states that the system is should implement functionality regarding creating and maintaining a budget. The system must support the user in creating precise budgets, by being able to accurately forecast the expected consumption of a building as stated in requirement F13.
\section{Selecting buildings for renovation}
Selecting buildings for renovation is a decision process with many political and economical parameters which would be impossible for one system to fully implement. However the system will be able to support the process by giving users the ability to quickly identify potential candidates. This is done by implementing requirement F11, which will make the system come up with various rankings of buildings, making the worst ranked buildings potential candidates for renovation. It will be considered to map the benchmarks of buildings to the national energy labeling scheme\footnote{ http://www.ens.dk/forbrug-besparelser/byggeriets-energiforbrug/energimaerkning }.
\section{Other requirements}
The environment in which the application is to be used, spawns quality requirements for the systems portability and scalability which needs to be meet in order to develop a successful application. 

The users from OM’s Energy and Maintenance department work both from their offices and from various locations when visiting buildings to inspect. This need is met by requirement XX, which addresses the portability of the system and states the application must work on a tablet. 

This project has focused on the municipality of Odense, but other municipalities in Denmark could potentially have similar requirements for an energy management system. Based on this, there is a requirement that the application is scalable. Another scenario is that the system is expanded to not only public buildings. In that case the number of buildings handled would be an order of magnitude more, and the system should be able to handle this. 
\section{Prioritization of requirements}
All requirements have been prioritized using the MoSCoW-prioritization technique. The prioritizations do not reflect the wishes of the client,  but are based on what is within this project's scope and boundaries. For instance a task management system is out of scope for this project and therefore requirements F16 and F17 have been prioritized as $WON’T$ have. If a finished product was to be made, these requirements are essential for the user and should therefore be implemented.
