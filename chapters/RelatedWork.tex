Within recent years numerous research papers have emerged showing successful applications of various methods to real word problems like detecting faults, benchmarking, classifying, predicting or recognizing patterns in the energy consumption of buildings. Most of the applied methods rely on theories rooted in the fields of statistics, machine learning and data mining to analyze and interpret the data. To successfully apply such methods to real word problems one must be familiar with the nature of the data, and have working knowledge of the methods being applied. This section will explore some of these methods and the current literature that applies them to various problems related to energy management.

\section{Decomposing time series}
The most visible pattern that emerge from time series data of building energy consumption is the daily rise and fall in consumption, which loops every 24 hours following the daily cycles of building users. Also a weekly cycle can be noticed, as different days of the week tend to have different consumption profiles. If looking at several years of data, the yearly cycle following the seasonal changes in environmental conditions can be seen. These cycles form the basis for the components that can be found when decomposing a time serie, and which could be used to reconstruct the original by combining the components. For instance considering the weekly cycles and yearly seasonal change, it would be possible to make a decomposition into a Cyclical Component, and a Seasonal Component. Additionally [REF wiki] suggests that the decomposition produces a Trend Component  and an Irregular Component. The Trend Component reflects the long term progression irrespective of any cycles and seasonality, while the Irregular Component is the noise in the original time serie which could not be explained with the other components. Figure XX shows an example of a decomposition: