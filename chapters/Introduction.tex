\pagenumbering{arabic}
The developed world’s dependency on fossil fuels to power growth and prosperity is becoming an increasingly unsustainable and unfeasible situation. The world will eventually run out of fossil fuels, and the dependency on it means that vital infrastructure is depending on politically unstable regions. What is perhaps even more concerning is the damage our environment suffers from our burning of fossil fuels. Energy efficiency is a major step away from the dependency on fossil fuels, and towards a society powered by economically and socially sustainable energy. 

In March 2012 a new Energy Agreement was reached in Denmark, with goals to have approximately 50\% of electricity consumption supplied by wind power, and more than 35\% of final energy consumption supplied from renewable energy sources in 2020. The final goal is to have 100\% renewable energy in the energy and transport sectors by 2050\footnote{Energy Efficiency Policies and Measures in Denmark 2012; Page 3.}.

Buildings in Denmark have over the last decades become more energy-efficient as a result of advances in engineering and the ongoing strengthening of requirements for new buildings. However, around 40\% of the total energy consumption  is used in buildings making them the largest contributor to energy usage in Denmark\footnote{http://www.kebmin.dk/klima-energi-bygningspolitik/dansk-klima-energi-bygningspolitik/byggeriet-danmark/bygningers}. Therefore retrofitting buildings with energy efficient technology is important, if Denmark is to fulfill the energy agreement and reach it’s goals. The field of Energy Informatics has recently attracted much attention as large energy efficiency improvements remains to be realized, by utilizing the field to better understand, predict and optimize energy consumption efficiency in buildings\footnote{Energy Informatics -DOI 10.1007/s12599-013-0304-2}. This field is a cornerstone for realizing the visions of smart buildings and smart cities. In southern Denmark, the Municipality of Odense (henceforth referred to as OM) has come one step closer to these visions, by putting data regarding energy consumption of approximately 550 public buildings on the roadmap to becoming Open Data\footnote{http://odensedataplatform.dk/}. This data contains hourly measurements of raw consumption for electricity, heating and water in a building. The availability of such data is a crucial step, but the raw data is of little use, unless it is transformed into useful information that can be represented in an intuitive manner to various stakeholders.

This project has worked with OM to find potential use cases of the consumption data, and explore what kind of useful information can be extracted from the available data. The aim was to develop a functioning prototype website capable of highlighting various trends, correlations, odd events, odd buildings etc. The scope for this project was not limited to researching and theoretically proving the applicability of various data processing methods, but included understanding and applying state of the art statistical methods to solve real problems for OM. Thus the challenges faced by this project were also those of a software development project, in which a software system suiting the clients’ need had to be developed and delivered. The project utilized an iterative development approach, to identify needs and problems related to building energy consumption in OM, design and evaluate prototypes and develop the final software solution.

The problems OM is facing and which the developed software attempts to solve, are rooted in issues relevant to the entire energy management domain. Those issues include how individuals and organizations can gain a better understanding of their energy consumption, and be motivated to save energy. How to identify odd consumption patterns and get a quicker response time to leakages. How to benchmark buildings to guide investments in renovations to yield optimal results. How to optimally support the workflows of building energy managers, while minimizing operation and maintenance efforts of such a system. How to more precisely predict energy consumption. Thus anyone working with those or related issues might be interested in the results and findings of this project.