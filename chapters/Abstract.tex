This study discus the application of statistical methods and machine learning in the field of buildings energy consumption. The project is developed in association the municipality of Odense, and the focus have been on creating value by solving existing problems.

The product is a server system and a website application which works a platform for obtaining information regarding the energy consumption in public buildings in the municipality of Odense. In addition, the product implements functionality enabling the user to maintain the system and data. The system is written in R, html, JavaScript and Java.


This report is a technical report, however the project have been conducted using a user centered iterative process model and therefore it contains a section of user evaluation.  