This report discusses the application of statistical methods and machine learning  to better understand and manage energy consumption in buildings. The report is the result of a project which was conducted  in association with the Municipality of Odense. The project’s focus was on creating a building energy management system which creates value and solves existing problems for the municipality.

The  resulting software system is a website  which presents information regarding  heat, water and electricity consumption in public buildings in the Municipality of Odense. The website enables the municipality's energy managers to easier maintain information about buildings, detect faults and leaks and compare buildings in respect to their consumption. The final system is a working prototype developed using a user centered iterative process and is written in R, html, JavaScript and Java.

This report will evaluate the statistical methods that were used, and the developed system’s performance for detecting faults and finding buildings with similar consumption patterns. Additionally the systems usability will be evaluated on basis of a user test with a representative from the municipalities Energy and Maintenance department.

The results show, that the system is capable of achieving good results in respect to fault detection and finding similar buildings. Additionally a user test shows that the system extracts useful information about consumption patterns, which can optimize the workflow of the employees in the municipalities Energy and Maintenance department and improve the decisions made.